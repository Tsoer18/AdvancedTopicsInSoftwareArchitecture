
In this project, we have managed to do a literature review to gain an understanding of the current state of the art in the field of assembly line production, which was expanded upon by doing detailed use cases and quality attribute scenarios. After this, a solution is proposed which features several tactics to try to increase some of the quality attributes for the system. This includes a heartbeat system for availabilty, a Github actions pipeline for deployability, dockerization of different parts of the system for scalability, the use of message busses and a commmon communication protocol (JSON) for interoperability and a queue system for performance. Then a experiment was setup to evaluate the system in terms of its availability. The experiment was to measure how long it would take for the system to alert a onsite system operator in case of a robot failing. The experiment showed that the average time the heartbeat system to detect a robot not responding was 2.45 seconds, which is more than the initial expected 2 seconds. 

In summary, different architectural strategies have been successfully applied to accommodate the quality attributes, whilst exploring the effects and trade-offs of the different architectures and technologies.

