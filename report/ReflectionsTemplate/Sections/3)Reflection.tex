
Our initial steps involved defining quality attributes, and the primary focus was on interoperability, availability, and deployability. The development of our use case, such as "User Orders a Tank" scenario, is showing the importance of what architecture is best suited for developing this system and what areas our system should cover. By using the theoretical knowledge with the practical challenges, we had to figure out how to use the complexities of Industry 4.0 system design, which has equipped us with a deeper understanding of the trade offs and considerations involved in developing a tank production and creating an adaptable software architecture. Looking back at the product we have produced, we can comfortably say that the system developed is a microservice system that achieves the quality attributes that was set for the system, in various extends.

Nonetheless, there were several things that did not go according to plan and should have been done differently.
As elaborated upon in the Future Work section of the report, there were some services that, due to time constraints, were left unfinished. Without the inventory service and the configuration service, the system as a whole is left unfinished. Though the services have been designed and the domain logic and responsibility areas of these services have been determined and is ready for implementation, given more time for development.

The time constraints stem from two areas.
Firstly, the research area being new to the team members, as we did not have a lot of prior knowledge about assembly lines, robots and Industry 4.0 systems as a whole, and therefore a lot of time was spent researching the area. Time constraints also stem from being unfamiliar with some of the technologies. Making sure that the system was interoperable and being careful to develop the system correctly. Learning C++ because we found that it had a faster runtime than most other programming languages\cite{876288} and being unfamiliar with MQTT messaging busses, also greatly added to the development time.
Secondly, the scope of the system being too broad. With the aforementioned lack of prior knowledge of the problem area, the scope we set for ourselves was broader than what we were able to accomplish in the given time frame. In retrospect we should have spent less time on, for an example, the UI and dedicated more time for system capabilities, such as the configuration and inventory services. Sacrificing some lesser aspects to achieve more technical functionality.

Having more time to thoroughly test the system would have been beneficial. Currently, we have only tested the response time of a robot that stops responding to the heartbeat monitor, which enforces our availability quality attribute. Nonetheless, further testing to check additional quality attributes and system functionality should have been performed. I.e. testing of use cases, interoperability, deployability, performance and scalability.

