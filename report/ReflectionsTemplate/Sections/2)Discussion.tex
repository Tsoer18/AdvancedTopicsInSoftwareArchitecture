
In the project there was a focus on availability, interoperability, and deployability and a set of predetermined requirements from the project description, which were as follows:
\begin{enumerate}
    \item Production software must be able to exchange and coordinate information to execute a production and change production
    \item Production software must run 24/7
    \item Production software must be continuously deployable
\end{enumerate}
To address the requirement of being able to exchange and coordinate information, MQTT buses were chosen. This resulted in most of our modules in the system being able to communicate using these message buses, which increased the interoperability of the system. This, however, was not done for the entirety of the system, as we had issues when integrating MQTT and our rails frontend, where the original idea was to send messages directly from the rails app to the message bus. This turned out a bit challenging, both in the sense that we could not configure the rails gem to handle MQTT connections properly, and that we realized that this direct link could have potential drawbacks, since we also should save orders in some sort of database, so that in case the message bus fails, the orders are still stored. This was solved by introducing a database and a connector module to handle extracting the orders from said database and integrating them with the JSON format used by the rest of the system.

With the focus on availability and the system being able to run 24/7, as well as deployability, a microservice architecture was chosen. Referring to the researched articles, a monolithic application would have been ideal if we were developing an operating system for the robots\cite{9659378}, but as we were developing the entire production line system around the robots, as well as the frontend and order system, the microservice architecture was more fitting\cite{8387665}. The reasoning for this is that a microservice architecture allows us to separate domain logic into individual independent services, which allows for separate deployability but also resiliency and availability. If an order has been created, scheduled, configured, and sent to the production system and then the entire system besides the production system breaks down, the production system will continue to operate.
Should a fault occur with the production system itself and one or more of the robots encounter an error, the heartbeat monitoring system will pick up on any faulty robots and the on-site responsible maintenance staff will be alerted as soon as possible, furthering the availability of the production system.

The solution falls short, in that it formally states that the "Production software must be able to exchange and coordinate information to execute a production and change production". In this scenario, we managed to incorporate the part about exchange and coordination of information within the system, but failed to create a system to handle the change of production. Our solution was designed so the management segment would contain a service called Configuration system, which would be responsible for setting the configurations of the robots before initiating the assembly line. Since our solution was to contain this aspect, but we did not have the time to implement it, this feature of the system would be in the near future of implementation as it only should require some configuration of the scheduling system and the simulated robots.

In addition to this, the integration with a front end did not go as smoothly as it could have, in that we wanted to make sure that the orders that were created persisted in the database, and gave the order table 2 column(isDone and isDeliveredToScheduler) to signify how far along in the system the order had gotten. Here the idea was that if the order was delivered to the scheduling module but the order did not get processed in a certain amount of time, the OrderCreator module would try to start the order again, assuming that somewhere in our system an error had occurred, which made it so our system got stuck and that the order was lost. We did not have the time to implement any meaningful measures for recovery from faults outside of the robots. This was because we figured that the essential part of a production line would be the robots. In summary for this segment, the solution that we have proposed has made some effort to increase availability and reliability but further efforts could be integrated in order to increase the availability even further.